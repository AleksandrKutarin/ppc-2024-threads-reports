\documentclass[a4paper, 14pt]{article}

\usepackage[english, russian]{babel}

\usepackage{tocloft}
\usepackage{tabularray}
\usepackage{hyperref}
\usepackage[utf8]{inputenc}
\usepackage[T2A]{fontenc}
\usepackage{indentfirst}
\usepackage{mathtext}
\frenchspacing

\usepackage{amsmath, amsfonts, amssymb, amsthm, mathtools}
\usepackage{icomma}

\renewcommand{\cftsecleader}{\cftdotfill{\cftdotsep}}
\newcommand{\n}{\par}

%%% Оформление страницы
\usepackage{extsizes}     % Возможность сделать 14-й шрифт
\usepackage{geometry}     % Простой способ задавать поля
\usepackage{setspace}     % Интерлиньяж
\usepackage{enumitem}     % Настройка окружений itemize и enumerate
\setlist{leftmargin=25pt} % Отступы в itemize и enumerate

\geometry{top=25mm}    % Поля сверху страницы
\geometry{bottom=30mm} % Поля снизу страницы
\geometry{left=20mm}   % Поля слева страницы
\geometry{right=20mm}  % Поля справа страницы

\setlength\parindent{15pt}        % Устанавливает длину красной строки 15pt
\linespread{1.3}                  % Коэффициент межстрочного интервала
\setlength{\parskip}{0.5em}      % Вертикальный интервал между абзацами
%\setcounter{secnumdepth}{0}      % Отключение нумерации разделов
%\setcounter{section}{-1}         % Нумерация секций с нуля
\usepackage{multicol}			  % Для текста в нескольких колонках
\usepackage{soulutf8}             % Модификаторы начертания
\begin{document}
	\thispagestyle{empty}
	\begin{center}
		
		МИНИСТЕРСТВО НАУКИ И ВЫСШЕГО ОБРАЗОВАНИЯ РОССИЙСКОЙ ФЕДЕРАЦИИ\n
		Федеральное государственное автономное образовательное учреждение высшего образования\n
		\textbf{«Национальный исследовательский Нижегородский государственный университет им. Н. И. Лобачевского»}\n
		Институт информационных технологий, математики и механики
		\vspace{1cm}
		
		ОТЧЁТ ПО ЛАБОРАТОРНЫМ РАБОТАМ
		
		ПО ДИСЦИПЛИНЕ
		
		\textbf{<<Параллельное программирование для систем с общей памятью>>}
		
		НА ТЕМУ
		
		\textbf{<<Умножение разреженных матриц. Элементы типа double. Формат хранения матрицы – строковый (CRS)>>}
	\end{center}
	\vspace{0.3cm}
	\begin{flushright}
		
		\textbf{Выполнил:}
		
		студент 3-го курса 
		
		гр. 3821Б1ФИ3 | МОСТ
		
		Сафаров Нурлан М.
	\end{flushright}
	
	\begin{flushright}
		\textbf{Проверил:}
		
		аспирант
		
		Нестеров А. Ю.
	\end{flushright}
	
	\begin{center}
		\vfill
		Нижний Новгород
		
		2024
	\end{center}
	\newpage
	\begin{center}\tableofcontents\end{center}
	\newpage
	\section*{\centering \textbf{Введение}}
	\addcontentsline{toc}{section}{Введение}
	
	В современном мире, где объемы данных постоянно растут, параллельные вычисления играют ключевую роль в обеспечении высокой производительности и эффективности алгоритмов обработки информации. Одним из важных направлений параллельного программирования является разработка алгоритмов для работы с разреженными матрицами, которые часто встречаются в прикладных областях, таких как машинное обучение, численное моделирование и анализ данных.
	
	В данном контексте, использование библиотек и инструментов для параллельных вычислений, таких как OpenMP, TBB (Threading Building Blocks) и стандартная библиотека потоков (STL), становится неотъемлемой частью разработки высокопроизводительных приложений. Эти инструменты предоставляют программистам удобные средства для создания параллельных программ, позволяя распределить вычислительную нагрузку на множество потоков и ядер процессора.
	
	\newpage
	\section*{\centering Постановка задачи}
	\addcontentsline{toc}{section}{Постановка задачи}
	\textbf{Цель работы:} реализовать и провести сравнительный анализ эффективности умножения разреженных матриц в формате CRS с использованием различных подходов к реализации. В данном контексте исследуются четыре варианта: последовательная реализация, параллельное выполнение с использованием OpenMP, TBB (Threading Building Blocks) и стандартной библиотеки потоков (STL). 
	
	\textbf{Задачи работы:}
	\vspace{-1em}
	\begin{itemize}[leftmargin=3em]
		\setlength\itemsep{0cm}
		\item разработать последовательную версию алгоритма умножения разреженных матриц;
		\item реализовать параллельную версию алгоритма с использованием OpenMP для распараллеливания вычислений на несколько потоков;
		\item создать параллельную версию алгоритма с помощью TBB (Threading Building Blocks) для эффективного управления потоками и задачами;
		\item разработать параллельную версию алгоритма с использованием стандартной библиотеки потоков (STL) для реализации параллельных вычислений;
		\item провести сравнительный анализ времени выполнения каждой версии алгоритма на различных наборах данных с разными характеристиками разреженности матриц.
	\end{itemize}
	
	\textbf{Использованное для реализации данных л/р оборудование и программное обеспечение:} 
	\vspace{-1em}
	\begin{itemize}[leftmargin=3em]
		\setlength\itemsep{0cm}
		\item Тип оборудования: ноутбук
		\item ОС: Windows 11 Домашняя
		\item Модель процессора: 12th Gen Intel(R) Core(TM) i7-12650H (2.70 GHz)
		\item Память: 16 ГБ (RAM) | 512 ГБ (ROM)
		\item Графический ускоритель: GeForce RTX 3060 для ноутбуков
		\item IDE: Visual Studio 2019
		\item Язык программирования: C/C++
	\end{itemize}
	
	
	\newpage
	\section*{\centering Описание алгоритма}
	\addcontentsline{toc}{section}{Описание алгоритма}
	\textbf{Последовательная версия:}
	\vspace{-1em}
	\begin{itemize}[leftmargin=3em]
		\setlength\itemsep{0cm}
		\item в данной версии алгоритма умножения разреженных матриц используется классический подход, основанный на структуре хранения разреженных матриц в формате CRS (Compressed Row Storage);
		\item алгоритм проходит по каждой строке исходной матрицы и вычисляет соответствующие элементы результирующей матрицы путем скалярного произведения строк и столбцов;
		\item результат сохраняется в виде разреженной матрицы в формате CRS.
	\end{itemize}
	
	\textbf{Параллельная версия с использованием OpenMP:}
	\vspace{-1em}
	\begin{itemize}[leftmargin=3em]
		\setlength\itemsep{0cm}
		\item в этой версии алгоритма используется директива OpenMP <<parallel for>>, чтобы распараллелить вычисления цикла по строкам исходной матрицы;
		\item каждый поток выполняет вычисления для своего диапазона строк, что позволяет эффективно использовать многопоточность для ускорения процесса умножения матриц.
	\end{itemize}
	
	\textbf{Параллельная версия с использованием TBB:}
	\vspace{-1em}
	\begin{itemize}[leftmargin=3em]
		\setlength\itemsep{0cm}
		\item в данной версии алгоритма используется библиотека TBB для создания и управления потоками;
		\item параллельность реализуется с помощью <<tbb::parallel\_for>>, которая автоматически распределяет выполнение цикла по разным потокам;
		\item каждый поток выполняет вычисления для своего диапазона строк матрицы, что позволяет эффективно использовать ресурсы многоядерных систем.
	\end{itemize}
	
	\textbf{Параллельная версия с использованием стандартной библиотеки потоков (STL):}
	\vspace{-1em}
	\begin{itemize}[leftmargin=3em]
		\setlength\itemsep{0cm}
		\item в этой версии алгоритма используются стандартные средства C++ для создания и управления потоками;
		\item вычисления для каждой строки матрицы параллельно выполняются в отдельных потоках с помощью класса <<std::thread>>;
		\item после завершения вычислений, результаты объединяются в единую результирующую матрицу;
	\end{itemize}
	
	
	\newpage
	\subsection*{\centering Описание программ}
	\addcontentsline{toc}{section}{Описание программ}
	Мы реализовали четыре версии умножения разреженных матриц: последовательную, параллельную с использованием OpenMP, параллельную с использованием TBB (Threading Building Blocks) и параллельную с использованием стандартной библиотеки потоков (STL). В каждой версии программы основное действие выполняется в методе run(), который является центральной частью алгоритма.
	
	Последовательная версия выполняет умножение матриц последовательно, вычисляя каждый элемент результирующей матрицы по формуле и сохраняя его в CRS формате.
	
	В параллельной версии с использованием OpenMP цикл по строкам исходной матрицы распараллеливается с помощью директивы <<\#pragma omp parallel for>>, что позволяет эффективно использовать доступные потоки для вычислений.
	
	Параллельная версия с использованием TBB использует функцию <<tbb::parallel\_for()>>, которая автоматически распределяет вычисления между потоками, обеспечивая эффективное использование многоядерных систем.
	
	В параллельной версии с использованием стандартной библиотеки потоков каждая строка матрицы обрабатывается в отдельном потоке с помощью класса <<std::thread>>, после чего результаты объединяются в единую результирующую матрицу.
	
	Реализация метода run() в последовательной версии программы
	\begin{verbatim}
		bool SparseMatrixMultiplicationCRS::run() {
			internal_order_test();
			
			std::vector<int> finalColumnIndexes;
			std::vector<int> finalPointers;
			std::vector<double> finalValues;
			int resultRows = X->numberOfRows;
			std::vector<std::vector<int>> localColumnIndexes(X->numberOfRows);
			std::vector<std::vector<double>> localValues(X->numberOfRows);
			
			int resultColumnIndexes = Y->numberOfRows;  // After transposing matrix Y
			
			for (int rOne = 0; rOne < X->numberOfRows; rOne++) {
				for (int rTwo = 0; rTwo < Y->numberOfRows; rTwo++) {
					int firstCurrentPointer = X->pointers[rOne];
					int secondCurrentPointer = Y->pointers[rTwo];
					int firstEndPointer = X->pointers[rOne + 1] - 1;
					int secondEndPointer = Y->pointers[rTwo + 1] - 1;
					double v = 0;
					
					while ((secondCurrentPointer <= secondEndPointer) && (firstCurrentPointer <= firstEndPointer)) {
						if (X->columnIndexes[firstCurrentPointer] <= Y->columnIndexes[secondCurrentPointer]) {
							if (X->columnIndexes[firstCurrentPointer] == Y->columnIndexes[secondCurrentPointer]) {
								v = v + (X->values[firstCurrentPointer]) * (Y->values[secondCurrentPointer]);
								secondCurrentPointer++;
								firstCurrentPointer++;
							} else {
								firstCurrentPointer++;
							}
						} else {
							secondCurrentPointer++;
						}
					}
					if (v != 0) {
						localValues[rOne].push_back(v);
						localColumnIndexes[rOne].push_back(rTwo);
					}
				}
			}
			int elementCounter = 0;
			finalPointers.push_back(elementCounter);
			
			for (int indRow = 0; indRow < X->numberOfRows; indRow++) {
				elementCounter = elementCounter + localColumnIndexes[indRow].size();
				finalColumnIndexes.insert(finalColumnIndexes.end(), localColumnIndexes[indRow].begin(),
				localColumnIndexes[indRow].end());
				finalValues.insert(finalValues.end(), localValues[indRow].begin(), localValues[indRow].end());
				finalPointers.push_back(elementCounter);
			}
			
			Z->numberOfColumns = resultColumnIndexes;
			Z->numberOfRows = resultRows;
			Z->values = finalValues;
			Z->columnIndexes = finalColumnIndexes;
			Z->pointers = finalPointers;
			
			return true;
		}
	\end{verbatim}
	
	Реализация метода run() в параллельной (OMP) версии программы
	\begin{verbatim}
		bool SparseMatrixMultiplicationCRS_OMP::run() {
			internal_order_test();
			
			std::vector<int> finalColumnIndexes;
			std::vector<int> finalPointers;
			std::vector<double> finalValues;
			int resultRows = X->numberOfRows;
			std::vector<std::vector<int>> localColumnIndexes(X->numberOfRows);
			std::vector<std::vector<double>> localValues(X->numberOfRows);
			
			int resultColumnIndexes = Y->numberOfRows;  // After transposing matrix Y
			
			omp_set_num_threads(4);
			#pragma omp parallel for
			for (int rOne = 0; rOne < X->numberOfRows; rOne++) {
				for (int rTwo = 0; rTwo < Y->numberOfRows; rTwo++) {
					int firstCurrentPointer = X->pointers[rOne];
					int secondCurrentPointer = Y->pointers[rTwo];
					int firstEndPointer = X->pointers[rOne + 1] - 1;
					int secondEndPointer = Y->pointers[rTwo + 1] - 1;
					double v = 0;
					
					while ((secondCurrentPointer <= secondEndPointer) && (firstCurrentPointer <= firstEndPointer)) {
						if (X->columnIndexes[firstCurrentPointer] <= Y->columnIndexes[secondCurrentPointer]) {
							if (X->columnIndexes[firstCurrentPointer] == Y->columnIndexes[secondCurrentPointer]) {
								v = v + (X->values[firstCurrentPointer]) * (Y->values[secondCurrentPointer]);
								secondCurrentPointer++;
								firstCurrentPointer++;
							} else {
								firstCurrentPointer++;
							}
						} else {
							secondCurrentPointer++;
						}
					}
					if (v != 0) {
						localValues[rOne].push_back(v);
						localColumnIndexes[rOne].push_back(rTwo);
					}
				}
			}
			int elementCounter = 0;
			finalPointers.push_back(elementCounter);
			
			for (int indRow = 0; indRow < X->numberOfRows; indRow++) {
				elementCounter = elementCounter + localColumnIndexes[indRow].size();
				finalColumnIndexes.insert(finalColumnIndexes.end(), localColumnIndexes[indRow].begin(),
				localColumnIndexes[indRow].end());
				finalValues.insert(finalValues.end(), localValues[indRow].begin(), localValues[indRow].end());
				finalPointers.push_back(elementCounter);
			}
			
			Z->numberOfColumns = resultColumnIndexes;
			Z->numberOfRows = resultRows;
			Z->values = finalValues;
			Z->columnIndexes = finalColumnIndexes;
			Z->pointers = finalPointers;
			
			return true;
		}
	\end{verbatim}
	
	Реализация метода run() в параллельной (TBB) версии программы
	\begin{verbatim}
		bool SparseMatrixMultiplicationCRS_TBB::run() {
			internal_order_test();
			
			std::vector<int> finalColumnIndexes;
			std::vector<int> finalPointers;
			std::vector<double> finalValues;
			int resultRows = X->numberOfRows;
			std::vector<std::vector<int>> localColumnIndexes(X->numberOfRows);
			std::vector<std::vector<double>> localValues(X->numberOfRows);
			
			int resultColumnIndexes = Y->numberOfRows;  // After transposing matrix Y
			
			int sizePart = 10;
			tbb::parallel_for(tbb::blocked_range<int>(0, X->numberOfRows, sizePart), [&](tbb::blocked_range<int> r) {
				for (int rOne = r.begin(); rOne != r.end(); ++rOne) {
					for (int rTwo = 0; rTwo < Y->numberOfRows; rTwo++) {
						int firstCurrentPointer = X->pointers[rOne];
						int secondCurrentPointer = Y->pointers[rTwo];
						int firstEndPointer = X->pointers[rOne + 1] - 1;
						int secondEndPointer = Y->pointers[rTwo + 1] - 1;
						double v = 0;
						
						while ((secondCurrentPointer <= secondEndPointer) && (firstCurrentPointer <= firstEndPointer)) {
							if (X->columnIndexes[firstCurrentPointer] <= Y->columnIndexes[secondCurrentPointer]) {
								if (X->columnIndexes[firstCurrentPointer] == Y->columnIndexes[secondCurrentPointer]) {
									v += (X->values[firstCurrentPointer]) * (Y->values[secondCurrentPointer]);
									secondCurrentPointer++;
									firstCurrentPointer++;
								} else {
									firstCurrentPointer++;
								}
							} else {
								secondCurrentPointer++;
							}
						}
						if (v != 0) {
							localValues[rOne].push_back(v);
							localColumnIndexes[rOne].push_back(rTwo);
						}
					}
				}
			});
			
			int elementCounter = 0;
			finalPointers.push_back(elementCounter);
			
			for (int indRow = 0; indRow < X->numberOfRows; indRow++) {
				elementCounter += localColumnIndexes[indRow].size();
				finalColumnIndexes.insert(finalColumnIndexes.end(), localColumnIndexes[indRow].begin(),
				localColumnIndexes[indRow].end());
				finalValues.insert(finalValues.end(), localValues[indRow].begin(), localValues[indRow].end());
				finalPointers.push_back(elementCounter);
			}
			
			Z->numberOfColumns = resultColumnIndexes;
			Z->numberOfRows = resultRows;
			Z->values = finalValues;
			Z->columnIndexes = finalColumnIndexes;
			Z->pointers = finalPointers;
			
			return true;
		}
	\end{verbatim}
	
	Реализация метода run() в параллельной (STL) версии программы
	\begin{verbatim}
		bool SparseMatrixMultiplicationCRS_STL::run() {
			internal_order_test();
			
			std::vector<int> finalColumnIndexes;
			std::vector<int> finalPointers;
			std::vector<double> finalValues;
			int resultRows = X->numberOfRows;
			std::vector<std::vector<int>> localColumnIndexes(X->numberOfRows);
			std::vector<std::vector<double>> localValues(X->numberOfRows);
			
			int resultColumnIndexes = Y->numberOfRows;  // After transposing matrix Y
			
			const int num_threads = 4;
			std::vector<std::thread> threads(num_threads);
			
			for (int i = 0; i < num_threads; ++i) {
				threads[i] = std::thread([&, i]() {
					for (int rOne = i; rOne < X->numberOfRows; rOne += num_threads) {
						for (int rTwo = 0; rTwo < Y->numberOfRows; rTwo++) {
							int firstCurrentPointer = X->pointers[rOne];
							int secondCurrentPointer = Y->pointers[rTwo];
							int firstEndPointer = X->pointers[rOne + 1] - 1;
							int secondEndPointer = Y->pointers[rTwo + 1] - 1;
							double v = 0;
							
							while ((secondCurrentPointer <= secondEndPointer) && (firstCurrentPointer <= firstEndPointer)) {
								if (X->columnIndexes[firstCurrentPointer] <= Y->columnIndexes[secondCurrentPointer]) {
									if (X->columnIndexes[firstCurrentPointer] == Y->columnIndexes[secondCurrentPointer]) {
										v += X->values[firstCurrentPointer] * Y->values[secondCurrentPointer];
										secondCurrentPointer++;
										firstCurrentPointer++;
									} else {
										firstCurrentPointer++;
									}
								} else {
									secondCurrentPointer++;
								}
							}
							if (v != 0) {
								localValues[rOne].push_back(v);
								localColumnIndexes[rOne].push_back(rTwo);
							}
						}
					}
				});
			}
			
			for (auto& thread : threads) {
				thread.join();
			}
			
			int elementCounter = 0;
			finalPointers.push_back(elementCounter);
			
			for (int indRow = 0; indRow < X->numberOfRows; indRow++) {
				elementCounter = elementCounter + localColumnIndexes[indRow].size();
				finalColumnIndexes.insert(finalColumnIndexes.end(), localColumnIndexes[indRow].begin(),
				localColumnIndexes[indRow].end());
				finalValues.insert(finalValues.end(), localValues[indRow].begin(), localValues[indRow].end());
				finalPointers.push_back(elementCounter);
			}
			
			Z->numberOfColumns = resultColumnIndexes;
			Z->numberOfRows = resultRows;
			Z->values = finalValues;
			Z->columnIndexes = finalColumnIndexes;
			Z->pointers = finalPointers;
			
			return true;
		}
	\end{verbatim}
	
	\newpage
	\section*{\centering Эксперименты и их результаты}
	\addcontentsline{toc}{section}{Эксперименты и их результаты}
	
	Во всех проведённых мной экспериментов была использована функция \textbf{createRandomMatrix()}. Она создаёт случайную матрицу заданного размера rows $\times$ columns, где каждый элемент матрицы заполняется случайным образом с вероятностью perc.
	Реализация метода createRandomMatrix()
	\begin{verbatim}
		std::vector<std::vector<double>> createRandomMatrix(int columns, int rows, double perc) {
			if (perc < 0 || perc > 1) {
				throw std::runtime_error("Wrong density. \n");
			}
			std::random_device mydev;
			std::vector<std::vector<double>> result = fillTheMatrixWithZeros(columns, rows);
			std::mt19937 gen(mydev());
			std::uniform_real_distribution<double> genP{0.0, 1.0};
			std::uniform_real_distribution<double> genVal{0.0, 25.0};
			for (int r = 0; r < rows; r++) {
				for (int c = 0; c < columns; c++) {
					if (genP(gen) <= perc) {
						result[r][c] = genVal(gen);
					}
				}
			}
			return result;
		}
	\end{verbatim}
	
	В ходе проведения эксперимента были выполнены пять тестов, включающих в себя матрицы разных размерностей. Количество потоков --- 4.  Результаты данных тестов представлены в след. таблице.
	
	\begin{center}
		\begin{tabular}{ ||c | c | c ||  }
			\hline Версия алгоритма & pipeline (в сек.) & task (в сек.)\\ 
			\hline Последовательная & 8.1262 & 8.0358 \\
			\hline OpenMP & 3.7390 & 3.6821 \\
			\hline TBB & 3.6994 & 3.6170 \\ 
			\hline STL & 3.8107 & 3.7997 \\ 
			\hline
		\end{tabular}\\[5mm]
	\end{center}
	
	\textbf{Анализ полученных данных.}
	
	\noindent Из представленных данных видно, что параллельные версии алгоритмов (OpenMP, TBB и STL) значительно превосходят последовательную версию по времени выполнения. Наименьшее время выполнения достигнуто с использованием TBB, а затем с небольшим отставанием идут версии с использованием OpenMP и STL. Однако различия во времени выполнения между этими тремя параллельными версиями незначительны. Это говорит о том, что все три библиотеки эффективно распределяют нагрузку между потоками и достигают схожих результатов.
	
	\textbf{Выводы:}
	\vspace{-1em}
	\begin{itemize}[leftmargin=3em]
		\setlength\itemsep{0cm}
		\item параллельные версии алгоритмов, реализованные с использованием библиотек OpenMP, TBB и STL, значительно превосходят последовательную версию по времени выполнения;
		\item наименьшее время выполнения достигнуто при использовании TBB, что указывает на эффективность данной библиотеки при распараллеливании вычислений;
		\item версии алгоритмов, использующие OpenMP и STL, также показывают хорошие результаты, хотя незначительно отстают от версии с использованием TBB;
		\item полученные результаты подтверждают эффективность параллельного программирования с использованием современных библиотек, что позволяет ускорить выполнение вычислительных задач на многоядерных системах.
	\end{itemize}
	
	\newpage
	\section*{\centering Заключение}
	\addcontentsline{toc}{section}{Заключение}
	
	Эти л/р были захватывающим погружением в мир параллельного программирования. Работая над четырьмя версиями алгоритма умножения разреженных матриц с использованием различных технологий, я получил богатый опыт в области параллельного программирования.
	
	Изучение каждой из технологий --- OpenMP, TBB и STL позволило мне лучше понять их особенности, преимущества и недостатки. Разработка и сравнение каждой версии алгоритма не только помогли мне углубить свои знания в области оптимизации производительности, но и дала понимание о том, как правильно выбирать подходящую технологию для конкретной задачи.
	
	Одной из ключевых целей было выполнение задачи с максимальной эффективностью при использовании каждой технологии. Полученные результаты показывают, что задача выполнена успешно: все реализации алгоритма дали значительное ускорение по сравнению с последовательной версией. Особенно важно отметить, что версия, основанная на TBB, продемонстрировала наилучшие результаты по времени выполнения, что говорит о ее высокой эффективности при распараллеливании задач.
	
	Этот проект дал мне ценный опыт и навыки в области параллельного программирования, а также позволил глубже понять, как использование правильных инструментов и технологий может значительно улучшить производительность и эффективность программного обеспечения.
	
	\newpage
	\section*{\centering Список литературы}
	\addcontentsline{toc}{section}{Список литературы}
	\begin{enumerate}[label={[\arabic*]}]
		\item Попов, А. Н. (2020). Параллельные алгоритмы в вычислительных системах. Москва: Издательство Технической Литературы
		\item Шастун, А. Е., \& Попов, А. Н. (2019). Эффективность алгоритмов сортировки в распределенных вычислительных средах. Журнал параллельных вычислений, 15(3), 112-127
		\item Лекции доцента кафедры ВВиСП ННГУ им. Н. И. Лобачевского, к.т.н. Сысоева А. В. по курсу <<Параллельное программирование для систем с общей памятью>>
		\item Матвиенко, С. Д., \& Новиков, К. Ф. (2018). Практическое руководство по алгоритмам и структурам данных. Нью-Йорк: Академическое издательство.
		\item Сафаров, Н. М. safarov-nm (GitHub Repository). URL:\href{https://github.com/safarov-nm}{https://github.com/safarov-nm}
	\end{enumerate}
	
	\newpage
	\section*{\centering Приложение}
	\addcontentsline{toc}{section}{Приложение}
	Файл sparse\_matmult\_crs.cpp
	\begin{verbatim}
		// Copyright 2024 Safarov Nurlan
		#include "seq/safarov_n_sparse_matmult_crs/include/sparse_matmult_crs.hpp"
		
		#include <algorithm>
		#include <cmath>
		#include <utility>
		#include <vector>
		
		SparseMatrixCRS::SparseMatrixCRS(int _numberOfColumns, int _numberOfRows, const std::vector<double>& _values,
		const std::vector<int>& _columnIndexes, const std::vector<int>& _pointers)
		: numberOfColumns(_numberOfColumns),
		numberOfRows(_numberOfRows),
		values(_values),
		columnIndexes(_columnIndexes),
		pointers(_pointers) {}
		
		SparseMatrixCRS::SparseMatrixCRS(int _numberOfColumns, int _numberOfRows) {
			numberOfColumns = _numberOfColumns;
			numberOfRows = _numberOfRows;
		}
		
		SparseMatrixCRS::SparseMatrixCRS(std::vector<std::vector<double>> matrix) {
			int indexCounter = 0;
			numberOfRows = matrix.size();
			numberOfColumns = matrix[0].size();
			pointers.push_back(indexCounter);
			for (int r = 0; r < numberOfRows; r++) {
				for (int c = 0; c < numberOfColumns; c++) {
					if (matrix[r][c] != 0) {
						values.push_back(matrix[r][c]);
						indexCounter++;
						columnIndexes.push_back(c);
					}
				}
				pointers.push_back(indexCounter);
			}
		}
		
		SparseMatrixCRS sparseMatrixTransposeCRS(const SparseMatrixCRS& object) {
			SparseMatrixCRS matrix;
			std::vector<std::vector<int>> locCVec(object.numberOfColumns);
			std::vector<std::vector<double>> locVecVal(object.numberOfColumns);
			matrix.numberOfColumns = object.numberOfRows;
			int elementCounter = 0;
			matrix.numberOfRows = object.numberOfColumns;
			
			for (int r = 0; r < object.numberOfRows; r++) {
				for (int index = object.pointers[r]; index < object.pointers[r + 1]; index++) {
					int cIndex = object.columnIndexes[index];
					locCVec[cIndex].push_back(r);
					locVecVal[cIndex].push_back(object.values[index]);
				}
			}
			matrix.pointers.push_back(elementCounter);
			for (int c = 0; c < object.numberOfColumns; c++) {
				for (size_t ktmp = 0; ktmp < locCVec[c].size(); ktmp++) {
					matrix.columnIndexes.push_back(locCVec[c][ktmp]);
					matrix.values.push_back(locVecVal[c][ktmp]);
				}
				elementCounter += locCVec[c].size();
				matrix.pointers.push_back(elementCounter);
			}
			return matrix;
		}
		
		bool SparseMatrixCRS::operator==(const SparseMatrixCRS& matrix) const {
			return (values == matrix.values) && (numberOfColumns == matrix.numberOfColumns) &&
			(columnIndexes == matrix.columnIndexes) && (numberOfRows == matrix.numberOfRows) &&
			(pointers == matrix.pointers);
		}
		
		std::vector<std::vector<double>> fillTheMatrixWithZeros(int columns, int rows) {
			std::vector<std::vector<double>> result(rows);
			for (int m = 0; m < rows; m++) {
				for (int n = 0; n < columns; n++) {
					result[m].push_back(0);
				}
			}
			return result;
		}
		
		std::vector<std::vector<double>> multiplyMatrices(std::vector<std::vector<double>> A,
		std::vector<std::vector<double>> B) {
			int p = B[0].size();
			int q = A.size();
			std::vector<std::vector<double>> resultMatrix = fillTheMatrixWithZeros(p, q);
			for (int rr = 0; rr < q; rr++) {
				for (int cc = 0; cc < p; cc++) {
					resultMatrix[rr][cc] = 0;
					for (size_t k = 0; k < A[0].size(); k++) {
						resultMatrix[rr][cc] += A[rr][k] * B[k][cc];
					}
				}
			}
			return resultMatrix;
		}
		
		std::vector<std::vector<double>> createRandomMatrix(int columns, int rows, double perc) {
			if (perc < 0 || perc > 1) {
				throw std::runtime_error("Wrong density. \n");
			}
			std::random_device mydev;
			std::vector<std::vector<double>> result = fillTheMatrixWithZeros(columns, rows);
			std::mt19937 gen(mydev());
			std::uniform_real_distribution<double> genP{0.0, 1.0};
			std::uniform_real_distribution<double> genVal{0.0, 25.0};
			for (int r = 0; r < rows; r++) {
				for (int c = 0; c < columns; c++) {
					if (genP(gen) <= perc) {
						result[r][c] = genVal(gen);
					}
				}
			}
			return result;
		}
		
		bool verifyCRSAttributes(const SparseMatrixCRS& object) {
			int nonZeroCount = object.values.size();
			auto check = size_t(nonZeroCount);
			
			if (object.pointers.size() != size_t(object.numberOfRows + 1)) {
				return false;
			}
			if (object.pointers[0] != 0) {
				return false;
			}
			if (object.values.size() != check || object.columnIndexes.size() != check ||
			object.pointers[object.numberOfRows] != nonZeroCount) {
				return false;
			}
			
			for (int i = 0; i < nonZeroCount; ++i) {
				if (object.columnIndexes[i] < 0 || object.columnIndexes[i] >= object.numberOfColumns) {
					return false;
				}
			}
			
			for (int j = 1; j <= object.numberOfRows; ++j) {
				if (object.pointers[j - 1] > object.pointers[j]) {
					return false;
				}
			}
			
			return true;
		}
		
		bool SparseMatrixMultiplicationCRS::validation() {
			internal_order_test();
			
			X = reinterpret_cast<SparseMatrixCRS*>(taskData->inputs[0]);
			Y = reinterpret_cast<SparseMatrixCRS*>(taskData->inputs[1]);
			Z = reinterpret_cast<SparseMatrixCRS*>(taskData->outputs[0]);
			
			if (X == nullptr || Y == nullptr || Z == nullptr) {
				return false;
			}
			
			if (!verifyCRSAttributes(*X) || !verifyCRSAttributes(*Y)) {
				return false;
			}
			
			if (taskData->inputs.size() != 2 || taskData->outputs.size() != 1 || !taskData->inputs_count.empty() ||
			!taskData->outputs_count.empty()) {
				return false;
			}
			
			if (taskData->inputs[0] == nullptr || taskData->inputs[1] == nullptr || taskData->outputs[0] == nullptr) {
				return false;
			}
			
			if (X->numberOfColumns != Y->numberOfRows) {
				return false;
			}
			
			return true;
		}
		
		bool SparseMatrixMultiplicationCRS::pre_processing() {
			internal_order_test();
			
			X = reinterpret_cast<SparseMatrixCRS*>(taskData->inputs[0]);
			Y = reinterpret_cast<SparseMatrixCRS*>(taskData->inputs[1]);
			Z = reinterpret_cast<SparseMatrixCRS*>(taskData->outputs[0]);
			
			*Y = sparseMatrixTransposeCRS(*Y);
			return true;
		}
		
		bool SparseMatrixMultiplicationCRS::run() {
			internal_order_test();
			
			std::vector<int> finalColumnIndexes;
			std::vector<int> finalPointers;
			std::vector<double> finalValues;
			int resultRows = X->numberOfRows;
			std::vector<std::vector<int>> localColumnIndexes(X->numberOfRows);
			std::vector<std::vector<double>> localValues(X->numberOfRows);
			
			int resultColumnIndexes = Y->numberOfRows;  // After transposing matrix Y
			
			for (int rOne = 0; rOne < X->numberOfRows; rOne++) {
				for (int rTwo = 0; rTwo < Y->numberOfRows; rTwo++) {
					int firstCurrentPointer = X->pointers[rOne];
					int secondCurrentPointer = Y->pointers[rTwo];
					int firstEndPointer = X->pointers[rOne + 1] - 1;
					int secondEndPointer = Y->pointers[rTwo + 1] - 1;
					double v = 0;
					
					while ((secondCurrentPointer <= secondEndPointer) && (firstCurrentPointer <= firstEndPointer)) {
						if (X->columnIndexes[firstCurrentPointer] <= Y->columnIndexes[secondCurrentPointer]) {
							if (X->columnIndexes[firstCurrentPointer] == Y->columnIndexes[secondCurrentPointer]) {
								v = v + (X->values[firstCurrentPointer]) * (Y->values[secondCurrentPointer]);
								secondCurrentPointer++;
								firstCurrentPointer++;
							} else {
								firstCurrentPointer++;
							}
						} else {
							secondCurrentPointer++;
						}
					}
					if (v != 0) {
						localValues[rOne].push_back(v);
						localColumnIndexes[rOne].push_back(rTwo);
					}
				}
			}
			int elementCounter = 0;
			finalPointers.push_back(elementCounter);
			
			for (int indRow = 0; indRow < X->numberOfRows; indRow++) {
				elementCounter = elementCounter + localColumnIndexes[indRow].size();
				finalColumnIndexes.insert(finalColumnIndexes.end(), localColumnIndexes[indRow].begin(),
				localColumnIndexes[indRow].end());
				finalValues.insert(finalValues.end(), localValues[indRow].begin(), localValues[indRow].end());
				finalPointers.push_back(elementCounter);
			}
			
			Z->numberOfColumns = resultColumnIndexes;
			Z->numberOfRows = resultRows;
			Z->values = finalValues;
			Z->columnIndexes = finalColumnIndexes;
			Z->pointers = finalPointers;
			
			return true;
		}
		
		bool SparseMatrixMultiplicationCRS::post_processing() {
			internal_order_test();
			
			return true;
		}
	\end{verbatim}
	
	Файл sparse\_matmult\_crs\_omp.cpp
	\begin{verbatim}
		// Copyright 2024 Safarov Nurlan
		#include "omp/safarov_n_sparse_matmult_crs/include/sparse_matmult_crs_omp.hpp"
		
		#include <omp.h>
		
		#include <algorithm>
		#include <cmath>
		#include <utility>
		#include <vector>
		
		SparseMatrixCRS::SparseMatrixCRS(int _numberOfColumns, int _numberOfRows, const std::vector<double>& _values,
		const std::vector<int>& _columnIndexes, const std::vector<int>& _pointers)
		: numberOfColumns(_numberOfColumns),
		numberOfRows(_numberOfRows),
		values(_values),
		columnIndexes(_columnIndexes),
		pointers(_pointers) {}
		
		SparseMatrixCRS::SparseMatrixCRS(int _numberOfColumns, int _numberOfRows) {
			numberOfColumns = _numberOfColumns;
			numberOfRows = _numberOfRows;
		}
		
		SparseMatrixCRS::SparseMatrixCRS(std::vector<std::vector<double>> matrix) {
			int indexCounter = 0;
			numberOfRows = matrix.size();
			numberOfColumns = matrix[0].size();
			pointers.push_back(indexCounter);
			for (int r = 0; r < numberOfRows; r++) {
				for (int c = 0; c < numberOfColumns; c++) {
					if (matrix[r][c] != 0) {
						values.push_back(matrix[r][c]);
						indexCounter++;
						columnIndexes.push_back(c);
					}
				}
				pointers.push_back(indexCounter);
			}
		}
		
		SparseMatrixCRS sparseMatrixTransposeCRS(const SparseMatrixCRS& object) {
			SparseMatrixCRS matrix;
			std::vector<std::vector<int>> locCVec(object.numberOfColumns);
			std::vector<std::vector<double>> locVecVal(object.numberOfColumns);
			matrix.numberOfColumns = object.numberOfRows;
			int elementCounter = 0;
			matrix.numberOfRows = object.numberOfColumns;
			
			for (int r = 0; r < object.numberOfRows; r++) {
				for (int index = object.pointers[r]; index < object.pointers[r + 1]; index++) {
					int cIndex = object.columnIndexes[index];
					locCVec[cIndex].push_back(r);
					locVecVal[cIndex].push_back(object.values[index]);
				}
			}
			matrix.pointers.push_back(elementCounter);
			for (int c = 0; c < object.numberOfColumns; c++) {
				for (size_t ktmp = 0; ktmp < locCVec[c].size(); ktmp++) {
					matrix.columnIndexes.push_back(locCVec[c][ktmp]);
					matrix.values.push_back(locVecVal[c][ktmp]);
				}
				elementCounter += locCVec[c].size();
				matrix.pointers.push_back(elementCounter);
			}
			return matrix;
		}
		
		bool SparseMatrixCRS::operator==(const SparseMatrixCRS& matrix) const {
			return (values == matrix.values) && (numberOfColumns == matrix.numberOfColumns) &&
			(columnIndexes == matrix.columnIndexes) && (numberOfRows == matrix.numberOfRows) &&
			(pointers == matrix.pointers);
		}
		
		std::vector<std::vector<double>> fillTheMatrixWithZeros(int columns, int rows) {
			std::vector<std::vector<double>> result(rows);
			for (int m = 0; m < rows; m++) {
				for (int n = 0; n < columns; n++) {
					result[m].push_back(0);
				}
			}
			return result;
		}
		
		std::vector<std::vector<double>> multiplyMatrices(std::vector<std::vector<double>> A,
		std::vector<std::vector<double>> B) {
			int p = B[0].size();
			int q = A.size();
			std::vector<std::vector<double>> resultMatrix = fillTheMatrixWithZeros(p, q);
			for (int rr = 0; rr < q; rr++) {
				for (int cc = 0; cc < p; cc++) {
					resultMatrix[rr][cc] = 0;
					for (size_t k = 0; k < A[0].size(); k++) {
						resultMatrix[rr][cc] += A[rr][k] * B[k][cc];
					}
				}
			}
			return resultMatrix;
		}
		
		std::vector<std::vector<double>> createRandomMatrix(int columns, int rows, double perc) {
			if (perc < 0 || perc > 1) {
				throw std::runtime_error("Wrong density. \n");
			}
			std::random_device mydev;
			std::vector<std::vector<double>> result = fillTheMatrixWithZeros(columns, rows);
			std::mt19937 gen(mydev());
			std::uniform_real_distribution<double> genP{0.0, 1.0};
			std::uniform_real_distribution<double> genVal{0.0, 25.0};
			for (int r = 0; r < rows; r++) {
				for (int c = 0; c < columns; c++) {
					if (genP(gen) <= perc) {
						result[r][c] = genVal(gen);
					}
				}
			}
			return result;
		}
		
		bool verifyCRSAttributes(const SparseMatrixCRS& object) {
			int nonZeroCount = object.values.size();
			auto check = size_t(nonZeroCount);
			
			if (object.pointers.size() != size_t(object.numberOfRows + 1)) {
				return false;
			}
			if (object.pointers[0] != 0) {
				return false;
			}
			if (object.values.size() != check || object.columnIndexes.size() != check ||
			object.pointers[object.numberOfRows] != nonZeroCount) {
				return false;
			}
			
			for (int i = 0; i < nonZeroCount; ++i) {
				if (object.columnIndexes[i] < 0 || object.columnIndexes[i] >= object.numberOfColumns) {
					return false;
				}
			}
			
			for (int j = 1; j <= object.numberOfRows; ++j) {
				if (object.pointers[j - 1] > object.pointers[j]) {
					return false;
				}
			}
			
			return true;
		}
		
		bool SparseMatrixMultiplicationCRS_OMP::validation() {
			internal_order_test();
			
			X = reinterpret_cast<SparseMatrixCRS*>(taskData->inputs[0]);
			Y = reinterpret_cast<SparseMatrixCRS*>(taskData->inputs[1]);
			Z = reinterpret_cast<SparseMatrixCRS*>(taskData->outputs[0]);
			
			if (X == nullptr || Y == nullptr || Z == nullptr) {
				return false;
			}
			
			if (!verifyCRSAttributes(*X) || !verifyCRSAttributes(*Y)) {
				return false;
			}
			
			if (taskData->inputs.size() != 2 || taskData->outputs.size() != 1 || !taskData->inputs_count.empty() ||
			!taskData->outputs_count.empty()) {
				return false;
			}
			
			if (taskData->inputs[0] == nullptr || taskData->inputs[1] == nullptr || taskData->outputs[0] == nullptr) {
				return false;
			}
			
			if (X->numberOfColumns != Y->numberOfRows) {
				return false;
			}
			
			return true;
		}
		
		bool SparseMatrixMultiplicationCRS_OMP::pre_processing() {
			internal_order_test();
			
			X = reinterpret_cast<SparseMatrixCRS*>(taskData->inputs[0]);
			Y = reinterpret_cast<SparseMatrixCRS*>(taskData->inputs[1]);
			Z = reinterpret_cast<SparseMatrixCRS*>(taskData->outputs[0]);
			
			*Y = sparseMatrixTransposeCRS(*Y);
			return true;
		}
		
		bool SparseMatrixMultiplicationCRS_OMP::run() {
			internal_order_test();
			
			std::vector<int> finalColumnIndexes;
			std::vector<int> finalPointers;
			std::vector<double> finalValues;
			int resultRows = X->numberOfRows;
			std::vector<std::vector<int>> localColumnIndexes(X->numberOfRows);
			std::vector<std::vector<double>> localValues(X->numberOfRows);
			
			int resultColumnIndexes = Y->numberOfRows;  // After transposing matrix Y
			
			omp_set_num_threads(4);
			#pragma omp parallel for
			for (int rOne = 0; rOne < X->numberOfRows; rOne++) {
				for (int rTwo = 0; rTwo < Y->numberOfRows; rTwo++) {
					int firstCurrentPointer = X->pointers[rOne];
					int secondCurrentPointer = Y->pointers[rTwo];
					int firstEndPointer = X->pointers[rOne + 1] - 1;
					int secondEndPointer = Y->pointers[rTwo + 1] - 1;
					double v = 0;
					
					while ((secondCurrentPointer <= secondEndPointer) && (firstCurrentPointer <= firstEndPointer)) {
						if (X->columnIndexes[firstCurrentPointer] <= Y->columnIndexes[secondCurrentPointer]) {
							if (X->columnIndexes[firstCurrentPointer] == Y->columnIndexes[secondCurrentPointer]) {
								v = v + (X->values[firstCurrentPointer]) * (Y->values[secondCurrentPointer]);
								secondCurrentPointer++;
								firstCurrentPointer++;
							} else {
								firstCurrentPointer++;
							}
						} else {
							secondCurrentPointer++;
						}
					}
					if (v != 0) {
						localValues[rOne].push_back(v);
						localColumnIndexes[rOne].push_back(rTwo);
					}
				}
			}
			int elementCounter = 0;
			finalPointers.push_back(elementCounter);
			
			for (int indRow = 0; indRow < X->numberOfRows; indRow++) {
				elementCounter = elementCounter + localColumnIndexes[indRow].size();
				finalColumnIndexes.insert(finalColumnIndexes.end(), localColumnIndexes[indRow].begin(),
				localColumnIndexes[indRow].end());
				finalValues.insert(finalValues.end(), localValues[indRow].begin(), localValues[indRow].end());
				finalPointers.push_back(elementCounter);
			}
			
			Z->numberOfColumns = resultColumnIndexes;
			Z->numberOfRows = resultRows;
			Z->values = finalValues;
			Z->columnIndexes = finalColumnIndexes;
			Z->pointers = finalPointers;
			
			return true;
		}
		
		bool SparseMatrixMultiplicationCRS_OMP::post_processing() {
			internal_order_test();
			
			return true;
		}
	\end{verbatim}
	
	Файл sparse\_matmult\_crs\_tbb.cpp
	\begin{verbatim}
		// Copyright 2024 Safarov Nurlan
		#include "tbb/safarov_n_sparse_matmult_crs/include/sparse_matmult_crs_tbb.hpp"
		
		#include <tbb/tbb.h>
		
		#include <algorithm>
		#include <cmath>
		#include <utility>
		#include <vector>
		
		SparseMatrixCRS::SparseMatrixCRS(int _numberOfColumns, int _numberOfRows, const std::vector<double>& _values,
		const std::vector<int>& _columnIndexes, const std::vector<int>& _pointers)
		: numberOfColumns(_numberOfColumns),
		numberOfRows(_numberOfRows),
		values(_values),
		columnIndexes(_columnIndexes),
		pointers(_pointers) {}
		
		SparseMatrixCRS::SparseMatrixCRS(int _numberOfColumns, int _numberOfRows) {
			numberOfColumns = _numberOfColumns;
			numberOfRows = _numberOfRows;
		}
		
		SparseMatrixCRS::SparseMatrixCRS(std::vector<std::vector<double>> matrix) {
			int indexCounter = 0;
			numberOfRows = matrix.size();
			numberOfColumns = matrix[0].size();
			pointers.push_back(indexCounter);
			for (int r = 0; r < numberOfRows; r++) {
				for (int c = 0; c < numberOfColumns; c++) {
					if (matrix[r][c] != 0) {
						values.push_back(matrix[r][c]);
						indexCounter++;
						columnIndexes.push_back(c);
					}
				}
				pointers.push_back(indexCounter);
			}
		}
		
		SparseMatrixCRS sparseMatrixTransposeCRS(const SparseMatrixCRS& object) {
			SparseMatrixCRS matrix;
			std::vector<std::vector<int>> locCVec(object.numberOfColumns);
			std::vector<std::vector<double>> locVecVal(object.numberOfColumns);
			matrix.numberOfColumns = object.numberOfRows;
			int elementCounter = 0;
			matrix.numberOfRows = object.numberOfColumns;
			
			for (int r = 0; r < object.numberOfRows; r++) {
				for (int index = object.pointers[r]; index < object.pointers[r + 1]; index++) {
					int cIndex = object.columnIndexes[index];
					locCVec[cIndex].push_back(r);
					locVecVal[cIndex].push_back(object.values[index]);
				}
			}
			matrix.pointers.push_back(elementCounter);
			for (int c = 0; c < object.numberOfColumns; c++) {
				for (size_t ktmp = 0; ktmp < locCVec[c].size(); ktmp++) {
					matrix.columnIndexes.push_back(locCVec[c][ktmp]);
					matrix.values.push_back(locVecVal[c][ktmp]);
				}
				elementCounter += locCVec[c].size();
				matrix.pointers.push_back(elementCounter);
			}
			return matrix;
		}
		
		bool SparseMatrixCRS::operator==(const SparseMatrixCRS& matrix) const {
			return (values == matrix.values) && (numberOfColumns == matrix.numberOfColumns) &&
			(columnIndexes == matrix.columnIndexes) && (numberOfRows == matrix.numberOfRows) &&
			(pointers == matrix.pointers);
		}
		
		std::vector<std::vector<double>> fillTheMatrixWithZeros(int columns, int rows) {
			std::vector<std::vector<double>> result(rows);
			for (int m = 0; m < rows; m++) {
				for (int n = 0; n < columns; n++) {
					result[m].push_back(0);
				}
			}
			return result;
		}
		
		std::vector<std::vector<double>> multiplyMatrices(std::vector<std::vector<double>> A,
		std::vector<std::vector<double>> B) {
			int p = B[0].size();
			int q = A.size();
			std::vector<std::vector<double>> resultMatrix = fillTheMatrixWithZeros(p, q);
			for (int rr = 0; rr < q; rr++) {
				for (int cc = 0; cc < p; cc++) {
					resultMatrix[rr][cc] = 0;
					for (size_t k = 0; k < A[0].size(); k++) {
						resultMatrix[rr][cc] += A[rr][k] * B[k][cc];
					}
				}
			}
			return resultMatrix;
		}
		
		std::vector<std::vector<double>> createRandomMatrix(int columns, int rows, double perc) {
			if (perc < 0 || perc > 1) {
				throw std::runtime_error("Wrong density. \n");
			}
			std::random_device mydev;
			std::vector<std::vector<double>> result = fillTheMatrixWithZeros(columns, rows);
			std::mt19937 gen(mydev());
			std::uniform_real_distribution<double> genP{0.0, 1.0};
			std::uniform_real_distribution<double> genVal{0.0, 25.0};
			for (int r = 0; r < rows; r++) {
				for (int c = 0; c < columns; c++) {
					if (genP(gen) <= perc) {
						result[r][c] = genVal(gen);
					}
				}
			}
			return result;
		}
		
		bool verifyCRSAttributes(const SparseMatrixCRS& object) {
			int nonZeroCount = object.values.size();
			auto check = size_t(nonZeroCount);
			
			if (object.pointers.size() != size_t(object.numberOfRows + 1)) {
				return false;
			}
			if (object.pointers[0] != 0) {
				return false;
			}
			if (object.values.size() != check || object.columnIndexes.size() != check ||
			object.pointers[object.numberOfRows] != nonZeroCount) {
				return false;
			}
			
			for (int i = 0; i < nonZeroCount; ++i) {
				if (object.columnIndexes[i] < 0 || object.columnIndexes[i] >= object.numberOfColumns) {
					return false;
				}
			}
			
			for (int j = 1; j <= object.numberOfRows; ++j) {
				if (object.pointers[j - 1] > object.pointers[j]) {
					return false;
				}
			}
			
			return true;
		}
		
		bool SparseMatrixMultiplicationCRS_TBB::validation() {
			internal_order_test();
			
			X = reinterpret_cast<SparseMatrixCRS*>(taskData->inputs[0]);
			Y = reinterpret_cast<SparseMatrixCRS*>(taskData->inputs[1]);
			Z = reinterpret_cast<SparseMatrixCRS*>(taskData->outputs[0]);
			
			if (X == nullptr || Y == nullptr || Z == nullptr) {
				return false;
			}
			
			if (!verifyCRSAttributes(*X) || !verifyCRSAttributes(*Y)) {
				return false;
			}
			
			if (taskData->inputs.size() != 2 || taskData->outputs.size() != 1 || !taskData->inputs_count.empty() ||
			!taskData->outputs_count.empty()) {
				return false;
			}
			
			if (taskData->inputs[0] == nullptr || taskData->inputs[1] == nullptr || taskData->outputs[0] == nullptr) {
				return false;
			}
			
			if (X->numberOfColumns != Y->numberOfRows) {
				return false;
			}
			
			return true;
		}
		
		bool SparseMatrixMultiplicationCRS_TBB::pre_processing() {
			internal_order_test();
			
			X = reinterpret_cast<SparseMatrixCRS*>(taskData->inputs[0]);
			Y = reinterpret_cast<SparseMatrixCRS*>(taskData->inputs[1]);
			Z = reinterpret_cast<SparseMatrixCRS*>(taskData->outputs[0]);
			
			*Y = sparseMatrixTransposeCRS(*Y);
			return true;
		}
		
		bool SparseMatrixMultiplicationCRS_TBB::run() {
			internal_order_test();
			
			std::vector<int> finalColumnIndexes;
			std::vector<int> finalPointers;
			std::vector<double> finalValues;
			int resultRows = X->numberOfRows;
			std::vector<std::vector<int>> localColumnIndexes(X->numberOfRows);
			std::vector<std::vector<double>> localValues(X->numberOfRows);
			
			int resultColumnIndexes = Y->numberOfRows;  // After transposing matrix Y
			
			int sizePart = 10;
			tbb::parallel_for(tbb::blocked_range<int>(0, X->numberOfRows, sizePart), [&](tbb::blocked_range<int> r) {
				for (int rOne = r.begin(); rOne != r.end(); ++rOne) {
					for (int rTwo = 0; rTwo < Y->numberOfRows; rTwo++) {
						int firstCurrentPointer = X->pointers[rOne];
						int secondCurrentPointer = Y->pointers[rTwo];
						int firstEndPointer = X->pointers[rOne + 1] - 1;
						int secondEndPointer = Y->pointers[rTwo + 1] - 1;
						double v = 0;
						
						while ((secondCurrentPointer <= secondEndPointer) && (firstCurrentPointer <= firstEndPointer)) {
							if (X->columnIndexes[firstCurrentPointer] <= Y->columnIndexes[secondCurrentPointer]) {
								if (X->columnIndexes[firstCurrentPointer] == Y->columnIndexes[secondCurrentPointer]) {
									v += (X->values[firstCurrentPointer]) * (Y->values[secondCurrentPointer]);
									secondCurrentPointer++;
									firstCurrentPointer++;
								} else {
									firstCurrentPointer++;
								}
							} else {
								secondCurrentPointer++;
							}
						}
						if (v != 0) {
							localValues[rOne].push_back(v);
							localColumnIndexes[rOne].push_back(rTwo);
						}
					}
				}
			});
			
			int elementCounter = 0;
			finalPointers.push_back(elementCounter);
			
			for (int indRow = 0; indRow < X->numberOfRows; indRow++) {
				elementCounter += localColumnIndexes[indRow].size();
				finalColumnIndexes.insert(finalColumnIndexes.end(), localColumnIndexes[indRow].begin(),
				localColumnIndexes[indRow].end());
				finalValues.insert(finalValues.end(), localValues[indRow].begin(), localValues[indRow].end());
				finalPointers.push_back(elementCounter);
			}
			
			Z->numberOfColumns = resultColumnIndexes;
			Z->numberOfRows = resultRows;
			Z->values = finalValues;
			Z->columnIndexes = finalColumnIndexes;
			Z->pointers = finalPointers;
			
			return true;
		}
		
		bool SparseMatrixMultiplicationCRS_TBB::post_processing() {
			internal_order_test();
			
			return true;
		}
	\end{verbatim}
	
	Файл sparse\_matmult\_crs\_stl.cpp
	\begin{verbatim}
		// Copyright 2024 Safarov Nurlan
		#include "stl/safarov_n_sparse_matmult_crs/include/sparse_matmult_crs_stl.hpp"
		
		#include <algorithm>
		#include <cmath>
		#include <thread>
		#include <utility>
		#include <vector>
		
		SparseMatrixCRS::SparseMatrixCRS(int _numberOfColumns, int _numberOfRows, const std::vector<double>& _values,
		const std::vector<int>& _columnIndexes, const std::vector<int>& _pointers)
		: numberOfColumns(_numberOfColumns),
		numberOfRows(_numberOfRows),
		values(_values),
		columnIndexes(_columnIndexes),
		pointers(_pointers) {}
		
		SparseMatrixCRS::SparseMatrixCRS(int _numberOfColumns, int _numberOfRows) {
			numberOfColumns = _numberOfColumns;
			numberOfRows = _numberOfRows;
		}
		
		SparseMatrixCRS::SparseMatrixCRS(std::vector<std::vector<double>> matrix) {
			int indexCounter = 0;
			numberOfRows = matrix.size();
			numberOfColumns = matrix[0].size();
			pointers.push_back(indexCounter);
			for (int r = 0; r < numberOfRows; r++) {
				for (int c = 0; c < numberOfColumns; c++) {
					if (matrix[r][c] != 0) {
						values.push_back(matrix[r][c]);
						indexCounter++;
						columnIndexes.push_back(c);
					}
				}
				pointers.push_back(indexCounter);
			}
		}
		
		SparseMatrixCRS sparseMatrixTransposeCRS(const SparseMatrixCRS& object) {
			SparseMatrixCRS matrix;
			std::vector<std::vector<int>> locCVec(object.numberOfColumns);
			std::vector<std::vector<double>> locVecVal(object.numberOfColumns);
			matrix.numberOfColumns = object.numberOfRows;
			int elementCounter = 0;
			matrix.numberOfRows = object.numberOfColumns;
			
			for (int r = 0; r < object.numberOfRows; r++) {
				for (int index = object.pointers[r]; index < object.pointers[r + 1]; index++) {
					int cIndex = object.columnIndexes[index];
					locCVec[cIndex].push_back(r);
					locVecVal[cIndex].push_back(object.values[index]);
				}
			}
			matrix.pointers.push_back(elementCounter);
			for (int c = 0; c < object.numberOfColumns; c++) {
				for (size_t ktmp = 0; ktmp < locCVec[c].size(); ktmp++) {
					matrix.columnIndexes.push_back(locCVec[c][ktmp]);
					matrix.values.push_back(locVecVal[c][ktmp]);
				}
				elementCounter += locCVec[c].size();
				matrix.pointers.push_back(elementCounter);
			}
			return matrix;
		}
		
		bool SparseMatrixCRS::operator==(const SparseMatrixCRS& matrix) const {
			return (values == matrix.values) && (numberOfColumns == matrix.numberOfColumns) &&
			(columnIndexes == matrix.columnIndexes) && (numberOfRows == matrix.numberOfRows) &&
			(pointers == matrix.pointers);
		}
		
		std::vector<std::vector<double>> fillTheMatrixWithZeros(int columns, int rows) {
			std::vector<std::vector<double>> result(rows);
			for (int m = 0; m < rows; m++) {
				for (int n = 0; n < columns; n++) {
					result[m].push_back(0);
				}
			}
			return result;
		}
		
		std::vector<std::vector<double>> multiplyMatrices(std::vector<std::vector<double>> A,
		std::vector<std::vector<double>> B) {
			int p = B[0].size();
			int q = A.size();
			std::vector<std::vector<double>> resultMatrix = fillTheMatrixWithZeros(p, q);
			for (int rr = 0; rr < q; rr++) {
				for (int cc = 0; cc < p; cc++) {
					resultMatrix[rr][cc] = 0;
					for (size_t k = 0; k < A[0].size(); k++) {
						resultMatrix[rr][cc] += A[rr][k] * B[k][cc];
					}
				}
			}
			return resultMatrix;
		}
		
		std::vector<std::vector<double>> createRandomMatrix(int columns, int rows, double perc) {
			if (perc < 0 || perc > 1) {
				throw std::runtime_error("Wrong density. \n");
			}
			std::random_device mydev;
			std::vector<std::vector<double>> result = fillTheMatrixWithZeros(columns, rows);
			std::mt19937 gen(mydev());
			std::uniform_real_distribution<double> genP{0.0, 1.0};
			std::uniform_real_distribution<double> genVal{0.0, 25.0};
			for (int r = 0; r < rows; r++) {
				for (int c = 0; c < columns; c++) {
					if (genP(gen) <= perc) {
						result[r][c] = genVal(gen);
					}
				}
			}
			return result;
		}
		
		bool verifyCRSAttributes(const SparseMatrixCRS& object) {
			int nonZeroCount = object.values.size();
			auto check = size_t(nonZeroCount);
			
			if (object.pointers.size() != size_t(object.numberOfRows + 1)) {
				return false;
			}
			if (object.pointers[0] != 0) {
				return false;
			}
			if (object.values.size() != check || object.columnIndexes.size() != check ||
			object.pointers[object.numberOfRows] != nonZeroCount) {
				return false;
			}
			
			for (int i = 0; i < nonZeroCount; ++i) {
				if (object.columnIndexes[i] < 0 || object.columnIndexes[i] >= object.numberOfColumns) {
					return false;
				}
			}
			
			for (int j = 1; j <= object.numberOfRows; ++j) {
				if (object.pointers[j - 1] > object.pointers[j]) {
					return false;
				}
			}
			
			return true;
		}
		
		bool SparseMatrixMultiplicationCRS_STL::validation() {
			internal_order_test();
			
			X = reinterpret_cast<SparseMatrixCRS*>(taskData->inputs[0]);
			Y = reinterpret_cast<SparseMatrixCRS*>(taskData->inputs[1]);
			Z = reinterpret_cast<SparseMatrixCRS*>(taskData->outputs[0]);
			
			if (X == nullptr || Y == nullptr || Z == nullptr) {
				return false;
			}
			
			if (!verifyCRSAttributes(*X) || !verifyCRSAttributes(*Y)) {
				return false;
			}
			
			if (taskData->inputs.size() != 2 || taskData->outputs.size() != 1 || !taskData->inputs_count.empty() ||
			!taskData->outputs_count.empty()) {
				return false;
			}
			
			if (taskData->inputs[0] == nullptr || taskData->inputs[1] == nullptr || taskData->outputs[0] == nullptr) {
				return false;
			}
			
			if (X->numberOfColumns != Y->numberOfRows) {
				return false;
			}
			
			return true;
		}
		
		bool SparseMatrixMultiplicationCRS_STL::pre_processing() {
			internal_order_test();
			
			X = reinterpret_cast<SparseMatrixCRS*>(taskData->inputs[0]);
			Y = reinterpret_cast<SparseMatrixCRS*>(taskData->inputs[1]);
			Z = reinterpret_cast<SparseMatrixCRS*>(taskData->outputs[0]);
			
			*Y = sparseMatrixTransposeCRS(*Y);
			return true;
		}
		
		bool SparseMatrixMultiplicationCRS_STL::run() {
			internal_order_test();
			
			std::vector<int> finalColumnIndexes;
			std::vector<int> finalPointers;
			std::vector<double> finalValues;
			int resultRows = X->numberOfRows;
			std::vector<std::vector<int>> localColumnIndexes(X->numberOfRows);
			std::vector<std::vector<double>> localValues(X->numberOfRows);
			
			int resultColumnIndexes = Y->numberOfRows;  // After transposing matrix Y
			
			const int num_threads = 4;
			std::vector<std::thread> threads(num_threads);
			
			for (int i = 0; i < num_threads; ++i) {
				threads[i] = std::thread([&, i]() {
					for (int rOne = i; rOne < X->numberOfRows; rOne += num_threads) {
						for (int rTwo = 0; rTwo < Y->numberOfRows; rTwo++) {
							int firstCurrentPointer = X->pointers[rOne];
							int secondCurrentPointer = Y->pointers[rTwo];
							int firstEndPointer = X->pointers[rOne + 1] - 1;
							int secondEndPointer = Y->pointers[rTwo + 1] - 1;
							double v = 0;
							
							while ((secondCurrentPointer <= secondEndPointer) && (firstCurrentPointer <= firstEndPointer)) {
								if (X->columnIndexes[firstCurrentPointer] <= Y->columnIndexes[secondCurrentPointer]) {
									if (X->columnIndexes[firstCurrentPointer] == Y->columnIndexes[secondCurrentPointer]) {
										v += X->values[firstCurrentPointer] * Y->values[secondCurrentPointer];
										secondCurrentPointer++;
										firstCurrentPointer++;
									} else {
										firstCurrentPointer++;
									}
								} else {
									secondCurrentPointer++;
								}
							}
							if (v != 0) {
								localValues[rOne].push_back(v);
								localColumnIndexes[rOne].push_back(rTwo);
							}
						}
					}
				});
			}
			
			for (auto& thread : threads) {
				thread.join();
			}
			
			int elementCounter = 0;
			finalPointers.push_back(elementCounter);
			
			for (int indRow = 0; indRow < X->numberOfRows; indRow++) {
				elementCounter = elementCounter + localColumnIndexes[indRow].size();
				finalColumnIndexes.insert(finalColumnIndexes.end(), localColumnIndexes[indRow].begin(),
				localColumnIndexes[indRow].end());
				finalValues.insert(finalValues.end(), localValues[indRow].begin(), localValues[indRow].end());
				finalPointers.push_back(elementCounter);
			}
			
			Z->numberOfColumns = resultColumnIndexes;
			Z->numberOfRows = resultRows;
			Z->values = finalValues;
			Z->columnIndexes = finalColumnIndexes;
			Z->pointers = finalPointers;
			
			return true;
		}
		
		bool SparseMatrixMultiplicationCRS_STL::post_processing() {
			internal_order_test();
			
			return true;
		}
	\end{verbatim}
	
\end{document}
